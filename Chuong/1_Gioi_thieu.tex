\documentclass[../DoAn.tex]{subfiles}
\begin{document}

\section{Đặt vấn đề}

Trong vài năm trở lại đây, khái niệm tài sản số và công nghệ chuỗi khối (blockchain) ngày càng được quan tâm rộng rãi trong cộng đồng công nghệ, đồng thời dần tiếp cận đến nhóm người dùng phổ thông. Cùng với xu hướng đó, tài sản số không thể thay thế (Non-Fungible Token — NFT) xuất hiện như một phương thức để gắn định danh và quyền sở hữu duy nhất cho từng đối tượng số, từ tranh vẽ, âm nhạc, vật phẩm trong trò chơi cho đến các bộ sưu tập sưu tầm. NFT cho phép người sáng tạo phát hành tác phẩm lên blockchain, còn người sưu tầm có thể mua, bán, trao đổi và tra cứu lịch sử sở hữu của từng tài sản.

Tuy nhiên, do NFT và các ứng dụng Web3 vẫn là lĩnh vực tương đối mới, trải nghiệm sử dụng đối với người dùng, đặc biệt là người mới bắt đầu, còn gặp nhiều trở ngại. Những trở ngại này đến từ đặc thù của công nghệ blockchain như việc phải sử dụng ví điện tử, thực hiện ký giao dịch, hiểu về phí gas và các chuẩn token, cũng như cách kết hợp giữa dữ liệu lưu trữ trên chuỗi (on-chain) và dữ liệu lưu trữ ngoài chuỗi (off-chain). Bên cạnh đó, người dùng cũng có nhu cầu rõ ràng về việc quản lý tài sản tập trung, theo dõi biến động giá trị và xem các thống kê liên quan theo cách trực quan, nhất quán để hỗ trợ quá trình sử dụng và ra quyết định.

Từ góc độ bài toán ứng dụng, có thể tổng hợp một số nhu cầu chính như sau:

\begin{itemize}
\item Cần quy trình thao tác rõ ràng, dễ tiếp cận đối với các chức năng cốt lõi như tạo bộ sưu tập, tạo NFT, niêm yết và thực hiện giao dịch mua bán.
\item Cần các màn hình quản lý giúp người sáng tạo theo dõi tình trạng và hiệu quả của bộ sưu tập, cũng như hoạt động của từng NFT theo thời gian.
\item Cần cơ chế hiển thị và tổng hợp dữ liệu như lịch sử giá, thống kê giao dịch, xu hướng và xếp hạng theo cách dễ hiểu và nhất quán giữa các trang chức năng.
\item Cần giải pháp lưu trữ metadata và hình ảnh NFT ổn định, dễ triển khai, đáp ứng nhu cầu truy cập nhanh trong môi trường ứng dụng web.
\item Cần hỗ trợ tốt cho các thao tác liên quan đến Web3 như kết nối ví, xác thực chữ ký và đồng bộ trạng thái tài sản sau khi giao dịch được ghi nhận trên blockchain.
\end{itemize}

Xuất phát từ các nhu cầu nêu trên, trong khuôn khổ đồ án tốt nghiệp, em lựa chọn nghiên cứu và xây dựng một website NFT marketplace tập trung vào các chức năng cốt lõi, đồng thời ưu tiên trải nghiệm sử dụng đơn giản và trực quan. Hệ thống hướng đến việc hỗ trợ người dùng tạo, mua bán, quản lý và theo dõi giá trị NFT theo thời gian, qua đó giúp giảm bớt rào cản khi tiếp cận Web3. Bên cạnh mục tiêu ứng dụng, đồ án cũng là cơ hội để em học hỏi và thực hành các nội dung quan trọng của lĩnh vực này, bao gồm tích hợp smart contract, xử lý giao dịch on-chain, tổ chức lưu trữ dữ liệu off-chain, cũng như thiết kế hệ thống theo hướng có khả năng mở rộng và dễ bảo trì.

\section{Mục tiêu và phạm vi đề tài}

\subsection{Mục tiêu của đề tài}
Trên cơ sở tham khảo các hệ thống hiện có và nhu cầu thực tế của người dùng khi tiếp cận NFT/Web3, đồ án hướng tới các mục tiêu chính sau:
\begin{itemize}
  \item Xây dựng một website NFT marketplace ở mức hoàn chỉnh theo các luồng cơ bản, cho phép người dùng kết nối ví, tạo NFT, niêm yết và thực hiện giao dịch mua/bán NFT.
  \item Hỗ trợ quản lý bộ sưu tập (collection) và tài sản (NFT) một cách tập trung, giúp người sáng tạo tạo collection, quản lý danh sách NFT thuộc collection và theo dõi trạng thái của từng NFT.
  \item Cung cấp chức năng tra cứu và theo dõi thông tin NFT/collection phục vụ nhu cầu sử dụng, bao gồm trang chi tiết, thông tin giá, lịch sử giao dịch và một số thống kê tổng quan phục vụ quan sát hoạt động cá nhân.
  \item Nghiên cứu và thực hành quy trình tích hợp ứng dụng Web2 với Web3, kết hợp giao dịch on-chain và dữ liệu off-chain nhằm đảm bảo hệ thống vừa đáp ứng tính minh bạch của blockchain, vừa đảm bảo khả năng hiển thị và truy vấn nhanh trên giao diện web.
\end{itemize}

\subsection{Phạm vi thực hiện}
Đồ án tập trung triển khai phiên bản đầu tiên với các phạm vi sau:
\begin{itemize}
  \item \textbf{Nền tảng:} xây dựng hệ thống theo mô hình 3 thành phần gồm giao diện web (frontend), dịch vụ xử lý nghiệp vụ (backend) và smart contract.
  \item \textbf{Môi trường blockchain:} triển khai và kiểm thử trên Binance Smart Chain (BSC) Testnet; định hướng có thể mở rộng để triển khai lên BSC Mainnet khi cần.
  \item \textbf{Chuẩn NFT và ví:} sử dụng chuẩn ERC-721; hỗ trợ kết nối ví Metamask để người dùng xác thực và thực hiện thao tác liên quan đến giao dịch.
  \item \textbf{Chức năng cốt lõi:}
  \begin{itemize}
    \item Quản lý người dùng theo mô hình đăng nhập bằng ví, quản lý hồ sơ và lịch sử giao dịch cá nhân.
    \item Tạo và quản lý collection; tạo NFT và gắn vào collection.
    \item Niêm yết và giao dịch mua/bán NFT thông qua smart contract; hiển thị trạng thái sở hữu và lịch sử giao dịch.
    \item Tra cứu, lọc và hiển thị danh sách NFT/collection; cung cấp một số thống kê và lịch sử giá phục vụ quan sát hoạt động.
  \end{itemize}
  \item \textbf{Lưu trữ dữ liệu:} kết hợp dữ liệu on-chain và dữ liệu off-chain; lưu trữ metadata và hình ảnh NFT trên dịch vụ lưu trữ đám mây để phục vụ truy cập nhanh và ổn định cho ứng dụng web.
\end{itemize}

\subsection{Các nội dung ngoài phạm vi của phiên bản đầu tiên}
\begin{itemize}
  \item Hỗ trợ đa chuỗi (multi-chain).
  \item Tích hợp các cổng thanh toán fiat on-ramp.
  \item Các cơ chế đấu giá phức tạp và các loại lệnh giao dịch nâng cao.
  \item Các tính năng DeFi nâng cao như staking hoặc lending.
\end{itemize}

\section{Định hướng giải pháp}

Để đạt được các mục tiêu trên, hệ thống được định hướng xây dựng theo các hướng chính sau:
\begin{itemize}
  \item \textbf{Phân tách on-chain và off-chain:}
  Smart contract chịu trách nhiệm các nghiệp vụ gắn với quyền sở hữu và giao dịch NFT (mint, tạo collection, niêm yết, mua/bán, thu phí). Backend và cơ sở dữ liệu tập trung quản lý dữ liệu phục vụ hiển thị và tra cứu như thông tin người dùng, metadata mở rộng, lịch sử giao dịch chi tiết, thống kê và dữ liệu tổng hợp. Dữ liệu được đồng bộ giữa on-chain và off-chain nhằm đảm bảo tính minh bạch nhưng vẫn đáp ứng hiệu năng khi truy vấn.

  \item \textbf{Kiến trúc hệ thống theo hướng dễ mở rộng và dễ bảo trì:}
  Hệ thống được tổ chức theo mô hình web hiện đại, tách biệt rõ ràng giữa giao diện (frontend), xử lý nghiệp vụ (backend) và lớp hợp đồng thông minh (smart contract). Các thành phần được thiết kế theo module/chức năng để thuận tiện phát triển, kiểm thử và mở rộng thêm các tính năng trong tương lai.

  \item \textbf{Tổ chức lưu trữ dữ liệu phù hợp với đặc thù NFT:}
  Dữ liệu dung lượng lớn (hình ảnh, tệp đính kèm) được lưu trữ trên dịch vụ lưu trữ bên ngoài, còn cơ sở dữ liệu lưu các thông tin phục vụ tra cứu và thống kê như NFT, collection, lịch sử giá và lịch sử giao dịch. Bên cạnh đó, hệ thống sử dụng cơ chế lưu đệm (caching) cho các dữ liệu truy cập thường xuyên nhằm cải thiện tốc độ phản hồi.

  \item \textbf{Tối ưu trải nghiệm người dùng:}
  Quy trình thao tác được thiết kế theo các bước rõ ràng, dễ theo dõi cho các chức năng chính như kết nối ví, tạo NFT, niêm yết và mua/bán. Hệ thống cung cấp các trang tổng quan giúp người dùng quan sát tài sản và hoạt động giao dịch của mình một cách trực quan.

  \item \textbf{Đảm bảo tính minh bạch và khả năng truy vết:}
  Các thông tin giao dịch quan trọng được liên kết với dữ liệu trên blockchain, cho phép người dùng kiểm tra lịch sử chuyển nhượng và thông tin giao dịch của từng NFT, từ đó tăng mức độ tin cậy đối với hệ thống.

  \item \textbf{Định hướng hiệu năng và khả năng mở rộng:}
  Hệ thống ưu tiên các giải pháp giúp truy vấn nhanh, giảm tải cho cơ sở dữ liệu và đảm bảo khả năng đáp ứng khi số lượng người dùng, NFT và giao dịch tăng lên trong tương lai.
\end{itemize}


\section{Bố cục đồ án}

Các nội dung còn lại của báo cáo đồ án tốt nghiệp được sắp xếp như sau:
\begin{itemize}
  \item \textbf{Chương 2 -- Khảo sát và phân tích yêu cầu}: trình bày bối cảnh và khảo sát các hệ thống liên quan, từ đó phân tích nhu cầu và xác định các yêu cầu chức năng, phi chức năng của hệ thống. Chương này đồng thời mô hình hóa yêu cầu bằng các biểu đồ và mô tả kịch bản sử dụng làm cơ sở cho các chương thiết kế và triển khai.

  \item \textbf{Chương 3 -- Cở sở lý thuyết và Công nghệ sử dụng}: giới thiệu cơ sở lý thuyết và các công nghệ, nền tảng được lựa chọn để xây dựng hệ thống, kèm theo lý do lựa chọn nhằm đảm bảo phù hợp với phạm vi, mục tiêu và đặc thù của bài toán NFT marketplace.

  \item \textbf{Chương 4 -- Thiết kế và triển khai hệ thống}: trình bày kiến trúc tổng thể, thiết kế cơ sở dữ liệu, thiết kế các thành phần chính và quá trình hiện thực các chức năng cốt lõi của hệ thống. Chương này cũng đưa ra kết quả kiểm thử và đánh giá mức độ đáp ứng yêu cầu đã đề ra.

  \item \textbf{Chương 5 -- Các giải pháp và đóng góp nổi bật}: tổng hợp các điểm nổi bật trong quá trình xây dựng hệ thống, bao gồm các quyết định thiết kế quan trọng, cách tiếp cận giải quyết một số vấn đề tiêu biểu và kết quả đạt được.

  \item \textbf{Chương 6 -- Kết luận và hướng phát triển}: tóm tắt kết quả thực hiện, đánh giá mức độ hoàn thành so với mục tiêu ban đầu, nêu hạn chế của hệ thống và đề xuất hướng phát triển trong tương lai.
\end{itemize}


Tóm lại, chương này giới thiệu bối cảnh, lý do lựa chọn đề tài, mục tiêu, phạm vi và định hướng giải pháp cho hệ thống website NFT marketplace. Các chương tiếp theo sẽ lần lượt đi sâu vào phần khảo sát, thiết kế, triển khai và đánh giá chi tiết hệ thống.

\end{document}
