\documentclass[../DoAn.tex]{subfiles}
\begin{document}

\begin{center}
    \Large{\textbf{ABSTRACT}}\\
\end{center}
\vspace{1cm}
In recent years, blockchain technology and digital assets have attracted increasing attention and have gradually reached a broader group of general users. Among them, non-fungible tokens (NFTs) have emerged as a mechanism for assigning unique identities and ownership to digital objects, enabling various applications in digital content creation and collectible assets. However, due to the inherent characteristics of blockchain and Web3 technologies, accessing and using NFT marketplaces still presents significant challenges, especially for new users. These challenges include the use of digital wallets, transaction signing, understanding transaction fees, as well as tracking and managing assets in a consistent and intuitive manner.

From this practical context, the thesis focuses on researching and developing an NFT marketplace website that aims to simplify the user experience while still supporting all essential business workflows. The system allows users to connect their wallets, create collections, mint NFTs, list assets for sale, and perform buying and selling transactions, as well as manage their assets and related information within a unified platform. Information that supports user decision-making, such as pricing, NFT status, transaction history, and basic statistics, is organized and presented in a clear and consistent way.

From a technical perspective, the project applies a hybrid on-chain and off-chain model, in which ownership-related operations and transactions are recorded on the blockchain, while data for presentation, search, and statistical analysis is managed off-chain to ensure efficient performance for web applications. Through the implementation of the system, the thesis not only aims to build a functional NFT marketplace capable of handling core workflows, but also enables the author to gain practical experience in integrating Web2 and Web3 technologies, organizing data, and designing a system that is maintainable and scalable.

\end{document}