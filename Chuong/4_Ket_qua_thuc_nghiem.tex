\documentclass[../DoAn.tex]{subfiles}
\usepackage{longtable}
\usepackage{array}
\usepackage{xurl} % cho URL tự ngắt dòng
\setlength{\LTleft}{0pt}
\setlength{\LTright}{0pt}
\usepackage{placeins}

\begin{document}

\section{Thiết kế kiến trúc}
\subsection{Lựa chọn kiến trúc phần mềm}

Kiến trúc phần mềm có thể được hiểu là cách tổ chức tổng thể của một hệ thống, bao gồm các thành phần chính, mối quan hệ giữa các thành phần và các nguyên tắc chi phối việc thiết kế và triển khai. Việc lựa chọn kiến trúc phù hợp giúp hệ thống có cấu trúc rõ ràng, dễ phát triển, dễ bảo trì và thuận lợi cho việc mở rộng trong tương lai. Trong thực tế, có nhiều phong cách kiến trúc được áp dụng tùy theo mục tiêu và bối cảnh như MVC/MVP trong ứng dụng hướng giao diện, kiến trúc phân lớp, Clean Architecture tập trung tách biệt business logic với chi tiết triển khai, hoặc Microservice hướng đến triển khai độc lập và khả năng mở rộng ở quy mô lớn.

Trong phạm vi đồ án này, em lựa chọn kiến trúc Modular Monolith. Đây là hướng tiếp cận trong đó hệ thống được triển khai như một ứng dụng thống nhất (monolith) nhằm đơn giản hóa quá trình triển khai và vận hành, đồng thời được chia thành các module theo nghiệp vụ (modular) để đảm bảo tính tổ chức, cô lập và dễ bảo trì. Mỗi module đóng vai trò quản lý tập trung các thành phần liên quan đến nghiệp vụ đó, nhờ vậy hạn chế tình trạng phụ thuộc chồng chéo và giúp mở rộng tính năng theo từng phần mà không ảnh hưởng đến các phần tính năng khác. Có thể xem Modular Monolith là sự cân bằng giữa monolith truyền thống và microservice. Không những phù hợp với điều kiện thời gian, nguồn lực và mục tiêu hoàn thiện các luồng nghiệp vụ của đồ án mà còn dễ dàng mở rộng nâng cấp khi hệ thống phát triển trong tương lai.

\begin{figure}[H]
    \centering
    \includegraphics[width=\textwidth]{Hinhve/modular_monolith.png}
    \caption{Ví dụ kiến trúc Modular monolith}
    \label{fig:Fig1}
\end{figure}

Dựa trên kiến trúc đã lựa chọn, hệ thống NFT Marketplace được thiết kế theo mô hình client--server kết hợp blockchain được biểu thị bằng hình vẽ dưới đây.

\begin{figure}[H]
    \centering
    \includegraphics[width=\textwidth]{Hinhve/kien_truc_he_thong.drawio.png}
    \caption{Kiến trúc của hệ thống NFT marketplace}
    \label{fig:Fig2}
\end{figure}

Ở phía client, người dùng thao tác trên giao diện web (Next.js/React) và tương tác với ví (Wallet Extension) để ký và gửi giao dịch blockchain. Các nghiệp vụ quan trọng như tạo, mua/bán NFT được thực hiện on-chain thông qua smart contract triển khai trên blockchain. Smart contract là thành phần nằm trên mạng blockchain, có nhiệm vụ thực thi logic giao dịch và phát sinh các sự kiện (events/logs) làm căn cứ xác minh giao dịch.

Ở phía server, backend được triển khai như một ứng dụng monolith và được chia thành các module nghiệp vụ chính gồm: Auth, User, NFT và History. Trong đó, Auth phụ trách xác thực và luồng kết nối ví; User quản lý thông tin hồ sơ người dùng; NFT xử lý nghiệp vụ liên quan NFT và trạng thái hiển thị; History ghi nhận các sự kiện và lịch sử giao dịch phục vụ hiển thị lịch sử và thống kê cơ bản. Với cách chia này, mỗi module có phạm vi trách nhiệm rõ ràng, dễ phát triển theo từng nhóm chức năng và thuận lợi cho việc bảo trì, mở rộng.

Về lưu trữ dữ liệu, hệ thống sử dụng PostgreSQL làm cơ sở dữ liệu chính cho dữ liệu nghiệp vụ, kết hợp Redis làm lớp cache để tối ưu các truy vấn đọc nhiều như danh sách NFT/collection, chi tiết NFT/collection và một số dữ liệu thống kê. Tài nguyên ảnh và metadata được lưu trên Amazon S3, đồng thời hệ thống có module upload hỗ trợ tải ảnh lên S3 để giảm tải cho backend và tối ưu tốc độ upload.

Một điểm thiết kế quan trọng của đồ án là cơ chế đồng bộ giữa dữ liệu on-chain và off-chain. Cụ thể, client thực hiện giao dịch blockchain qua ví và nhận txHash. Sau đó, client gọi API lên backend để lưu giữ liệu. Backend sẽ truy vấn blockchain thông qua RPC provider để kiểm tra giao dịch đã thành công hay chưa, có đúng smart contract và dữ liệu trong event có khớp với yêu cầu nghiệp vụ hay không. Chỉ khi xác minh hợp lệ, backend mới cập nhật dữ liệu vào PostgreSQL, ghi nhận lịch sử vào History module và cập nhật cache Redis khi cần. Cách làm này giúp giảm sai lệch trạng thái hiển thị off-chain so với trạng thái thực tế on-chain, đồng thời nâng cao tính tin cậy của hệ thống.

\subsection{Thiết kế tổng quan}
Với thiết kế theo hướng tách thành các module, mỗi module sẽ là một package riêng.
\begin{figure}[H]
    \centering
    \includegraphics[width=\textwidth]{Hinhve/bieu_do_goi_tong_quan.drawio.png}
    \caption{Biểu đồ tổng quan gói}
    \label{fig:Fig3}
\end{figure}
Dưới đây là mô tả sơ lược về từng package:

Auth: Quản lý xác thực người dùng dựa trên ví. Package này chịu trách nhiệm kiểm tra chữ ký, tạo/duy trì token phiên đăng nhập.

User: Quản lý thông tin người dùng trong hệ thống. Package này lưu trữ và cập nhật hồ sơ người dùng.

NFT: Chịu trách nhiệm tạo, quản lý trạng thái và mua bán NFT.

Upload: Quản lý upload và lưu trữ tài nguyên đa phương tiện.

History: Ghi nhận và truy vấn lịch sử hoạt động/giao dịch. Package này lưu các sự kiện khi phát sinh giao dịch trên smart contract, phục vụ màn hình lịch sử, thống kê cơ bản theo thời gian.

SmartContract: Giúp hệ thống tương tác với blockchain và smart contract.

\subsection{Thiết kế chi tiết gói}
Các package được tổ chức theo hướng module nên kiến trúc mỗi gói là tương tự nhau, dưới đây là chi tiết về package

\begin{figure}[H]
    \centering
    \includegraphics[width=\textwidth]{Hinhve/bieu_do_goi_chi_tiet.drawio.png}
    \caption{Biểu đồ chi tiết gói}
    \label{fig:Fig4}
\end{figure}

Controller: Đảm nhận vai trò tiếp nhận và xử lý các yêu cầu từ phía client thông qua các API. Nhiệm vụ chính của gói này là định tuyến yêu cầu đến đúng chức năng, kiểm tra/chuẩn hoá dữ liệu đầu vào và trả về phản hồi theo định dạng thống nhất. Controller không tập trung triển khai nghiệp vụ, mà đóng vai trò kết nối giữa client và hệ thống.

Service: Là nơi triển khai các nghiệp vụ cốt lõi của từng module. Gói này chịu trách nhiệm điều phối luồng xử lý: gọi các lớp truy cập dữ liệu, kết hợp các bước xử lý nghiệp vụ, áp dụng các quy tắc/điều kiện, và phối hợp với các module liên quan khi cần. Có thể xem service là tầng xử lý chính giúp đảm bảo nghiệp vụ được thực thi đúng và nhất quán.

Repository: Phụ trách tương tác với nguồn dữ liệu và các thành phần hạ tầng lưu trữ. Mục đích chính của gói này là đóng gói các thao tác đọc/ghi dữ liệu, giúp tầng service không phụ thuộc trực tiếp vào chi tiết triển khai như ORM, câu truy vấn SQL hay cơ chế lưu trữ cụ thể. Nhờ đó, hệ thống dễ bảo trì và dễ thay đổi công nghệ lưu trữ khi cần.

Entity: Mục đích của gói này là chuẩn hoá cấu trúc dữ liệu, quan hệ giữa các đối tượng, và làm nền tảng để các tầng phía trên thao tác một cách thống nhất, tránh việc xử lý dữ liệu rời rạc, thiếu nhất quán.

Utils: Chứa các thành phần hỗ trợ dùng chung trong phạm vi module. Mục đích của utils là giảm trùng lặp mã nguồn, tăng khả năng tái sử dụng và giữ cho service tập trung vào nghiệp vụ chính.

DTO: Định nghĩa các cấu trúc dữ liệu dùng để trao đổi giữa client và server. Mục đích chính là đảm bảo cấu trúc dữ liệu rõ ràng, hỗ trợ kiểm tra tính hợp lệ của dữ liệu đầu vào và chuẩn hoá dữ liệu đầu ra. Việc tách DTO khỏi entity giúp tránh lộ cấu trúc nội bộ và tăng tính ổn định của API.

Constant: Tập hợp các hằng số và giá trị cấu hình logic được sử dụng xuyên suốt module. Mục đích của gói này là hạn chế việc cấu hình mã nguồn rải rác trong hệ thống, đảm bảo tính nhất quán và giúp việc thay đổi/điều chỉnh trở nên thuận tiện, an toàn hơn.

\section{Thiết kế chi tiết}
\subsection{Thiết kế giao diện}
Phần giao diện của hệ thống NFT Marketplace được thiết kế theo hướng responsive nhằm đảm bảo khả năng sử dụng tốt trên cả máy tính (desktop/laptop) và thiết bị di động (mobile). Mục tiêu của thiết kế là tạo trải nghiệm nhất quán, dễ thao tác, đồng thời hỗ trợ người dùng thực hiện các chức năng chính như connect wallet, tạo NFT, mua/bán NFT trên nhiều kích thước màn hình khác nhau.

\textbf{Thông tin về màn hình mà ứng dụng hướng tới}

Hệ thống hướng đến các thiết bị phổ biến trên thị trường hiện nay, bao gồm máy tính và điện thoại thông minh. Các độ phân giải dưới đây được lựa chọn do có tần suất sử dụng cao và đại diện tốt cho phần lớn thiết bị người dùng.

\begin{table}[H]
\centering
\renewcommand{\arraystretch}{1.2}
\begin{tabular}{|c|l|l|l|}
\hline
\textbf{STT} & \textbf{Loại thiết bị} & \textbf{Kích thước phổ biến} & \textbf{Độ phân giải mục tiêu} \\
\hline
1 & Máy tính để bàn / Laptop & 13'' -- 15.6'', 21'' trở lên & 1366$\times$768; 1920$\times$1080; 2560$\times$1440 \\
\hline
2 & Điện thoại thông minh & 5'' -- 6.9'' & 1280$\times$720; 1920$\times$1080 \\
\hline
\end{tabular}
\caption{Bảng mô tả thông tin về màn hình mà hệ thống hướng tới}
\label{tab:ui-target-screens}
\end{table}

\textbf{Chuẩn hoá thiết kế giao diện}

Để đảm bảo tính thống nhất giữa các màn hình và giảm sai lệch trong quá trình triển khai, đồ án đưa ra các quy tắc chuẩn hoá giao diện được thiết kế theo bố cục linh hoạt, thay đổi theo độ rộng màn hình (breakpoint). Các breakpoint được sử dụng theo nhóm phổ biến để đảm bảo tương thích tốt:

\begin{table}[H]
\centering
\renewcommand{\arraystretch}{1.2}
\begin{tabular}{|l|c|p{8.2cm}|}
\hline
\textbf{Nhóm màn hình} & \textbf{Độ rộng (tham khảo)} & \textbf{Nguyên tắc bố cục} \\
\hline
Mobile & $\leq$ 576px & Bố cục một cột, ưu tiên nội dung chính, nút hành động rõ ràng, thao tác chạm thuận tiện \\
\hline
Tablet nhỏ & 577--768px & Hai cột đơn giản, giảm mật độ thông tin \\
\hline
Tablet lớn / Laptop & 769--1024px & Ba cột, bắt đầu hiển thị thêm khu vực lọc/sắp xếp khi phù hợp \\
\hline
Desktop & 1025--1440px & Nhiều cột, danh sách NFT hiển thị dạng lưới (grid) để tăng khả năng quan sát \\
\hline
Desktop lớn & $>$ 1440px & Tăng khoảng trắng, giữ chiều rộng nội dung hợp lý, tránh dàn trải quá rộng \\
\hline
\end{tabular}
\caption{Các breakpoint và nguyên tắc bố cục responsive}
\label{tab:ui-breakpoints}
\end{table}

Nguyên tắc tổng quát:

Trên desktop/laptop, danh sách NFT và collection hiển thị theo dạng lưới (grid) để tăng khả năng quan sát.

Trên mobile, giao diện chuyển sang một cột theo chiều dọc; thẻ NFT được thiết kế đủ lớn để dễ thao tác.

Các thông tin trọng tâm như hình ảnh NFT, tên NFT/collection, giá, trạng thái và nút hành động được ưu tiên đặt ở vùng dễ nhìn nhằm giảm số thao tác cuộn.

\textbf{Chuẩn hoá phối màu và hiển thị chữ}

Giao diện sử dụng tông màu xám và trắng làm nền chủ đạo nhằm tạo cảm giác hiện đại, tối giản và làm nổi bật nội dung hình ảnh. Màu xanh nước biển được sử dụng làm màu nhấn cho các hành động quan trọng và các thành phần cần thu hút sự chú ý.

Quy ước màu chữ:

Chữ màu xám cho nội dung thông thường và mô tả phụ.

Chữ màu xanh nước biển cho nội dung cần nhấn mạnh như liên kết, trạng thái, hoặc điểm nổi bật.

Chữ màu trắng dùng trên nền tối hoặc trên nút hành động để đảm bảo độ tương phản.

\textbf{Chuẩn hoá nút bấm và điều khiển}

Hệ thống chuẩn hoá 2 loại nút chính:

Primary Button dùng cho hành động quan trọng như: \textit{Connect Wallet}, \textit{Create NFT}.
\textbf{(2) Secondary Button} dùng cho hành động phụ/hỗ trợ như: \textit{Cancel}.

Các nút được thiết kế thống nhất về kiểu dáng và trạng thái hiển thị:
(1) Trạng thái bình thường: hiển thị rõ nội dung, dễ nhận biết.
(2) Trạng thái hover/active (desktop): thay đổi sắc độ/viền để phản hồi thao tác.
(3) Trạng thái disabled: giảm độ nổi bật để biểu thị không thể thao tác.
(4) Trạng thái loading: hiển thị tiến trình khi hệ thống đang xử lý các thao tác như xác nhận giao dịch hoặc tải dữ liệu.

\textbf{Minh họa thiết kế giao diện(đoạn này cần thêm ảnh vào, sau đọc lại nhớ bổ sung)}

\subsection{Thiết kế lớp}

\begin{figure}[H]
    \centering
    \includegraphics[width=\textwidth]{Hinhve/bieu_do_lop_auth.drawio.png}
    \caption{Biểu đồ Lớp Auth service}
    \label{fig:Fig5}
\end{figure}

Lớp AuthService được sử dụng để thực hiện xác thực người dùng dựa trên địa chỉ ví và chữ ký (signature). Khi người dùng gửi yêu cầu đăng nhập, hệ thống chuẩn hoá địa chỉ ví để đảm bảo tính nhất quán dữ liệu. Nếu request có kèm chữ ký và thông điệp ký, backend sẽ tiến hành xác minh chữ ký nhằm đảm bảo người dùng thực sự sở hữu ví đó. Sau khi xác thực hợp lệ, hệ thống tìm hoặc tự động tạo mới tài khoản theo địa chỉ ví, sau đó phát hành JWT để người dùng sử dụng trong các request tiếp theo. Lớp cũng hỗ trợ xác thực theo chuẩn SIWE và cung cấp các chức năng phụ trợ như tạo nonce và truy vấn hồ sơ người dùng.

\begin{figure}[H]
    \centering
    \includegraphics[width=\textwidth]{Hinhve/bieu_do_lop_nft.drawio.png}
    \caption{Biểu đồ Lớp NFT service}
    \label{fig:Fig6}
\end{figure}

Lớp NFTService là lớp nghiệp vụ trung tâm của module NFT. Lớp này chịu trách nhiệm quản lý vòng đời của NFT trong hệ thống off-chain (tạo NFT, truy vấn, cập nhật, xoá), đồng thời xử lý các luồng nghiệp vụ gắn với blockchain như mint, list/unlist và purchase. Điểm quan trọng trong thiết kế là hệ thống không ghi nhận kết quả giao dịch chỉ dựa trên dữ liệu từ client, mà sử dụng txHash để truy vấn transaction và xác minh event phát sinh trên blockchain. Khi dữ liệu xác minh hợp lệ, backend mới cập nhật trạng thái NFT trong cơ sở dữ liệu và ghi nhận lịch sử giao dịch (event) để phục vụ các chức năng của history. Cách tiếp cận này giúp dữ liệu off-chain đồng bộ và có cơ sở tin cậy từ dữ liệu on-chain.

\textbf{Biểu đồ trinh tự kết nối ví}

\textbf{Biểu đồ trinh mua NFT}

\subsection{Thiết kế cơ sở dữ liệu}

Thiết kế sơ đồ thực thể liên kết:

\begin{figure}[H]
    \centering
    \includegraphics[width=\textwidth]{Hinhve/database.png}
    \caption{Sơ đồ thực thể liên kết}
    \label{fig:Fig7}
\end{figure}

Đặc tả bảng User:

\begin{longtable}{|l|l|p{9cm}|}
\hline
\textbf{Trường} & \textbf{Kiểu dữ liệu} & \textbf{Mô tả} \\ \hline
\endfirsthead
\hline
\textbf{Trường} & \textbf{Kiểu dữ liệu} & \textbf{Mô tả} \\ \hline
\endhead

id & BIGINT & Khóa chính (PK) định danh người dùng. \\ \hline
wallet\_address & VARCHAR(42) & Địa chỉ ví (0x...), dùng để đăng nhập/định danh on-chain. \\ \hline
username & VARCHAR(100) & Tên hiển thị của người dùng trên hệ thống. \\ \hline
role & VARCHAR(30) & Vai trò người dùng (ví dụ: user/admin). \\ \hline
description & TEXT & Mô tả/bio của người dùng. \\ \hline
avatar & TEXT & URL ảnh đại diện. \\ \hline
facebook\_url & TEXT & Liên kết Facebook (nếu có). \\ \hline
instagram\_url & TEXT & Liên kết Instagram (nếu có). \\ \hline
youtube\_url & TEXT & Liên kết Youtube (nếu có). \\ \hline
banner & TEXT & URL ảnh bìa (banner). \\ \hline
created\_at & TIMESTAMP & Thời điểm tạo bản ghi. \\ \hline
updated\_at & TIMESTAMP & Thời điểm cập nhật gần nhất. \\ \hline

\caption{Chi tiết bảng người dùng}
\label{tab:user-table}
\end{longtable}

Đặc tả bảng Collection:

\begin{longtable}{|l|l|p{9cm}|}
\hline
\textbf{Trường} & \textbf{Kiểu dữ liệu} & \textbf{Mô tả} \\ \hline
\endfirsthead
\hline
\textbf{Trường} & \textbf{Kiểu dữ liệu} & \textbf{Mô tả} \\ \hline
\endhead

id & BIGINT & Khóa chính (PK) định danh collection. \\ \hline
name & VARCHAR(150) & Tên collection. \\ \hline
description & TEXT & Mô tả collection. \\ \hline
creator\_id & BIGINT & Khóa ngoại (FK) \(\rightarrow\) users.id, người tạo collection. \\ \hline
chain\_id & INT & Mã mạng blockchain (ví dụ: 1, 56, 97...). \\ \hline
contract\_address & VARCHAR(42) & Địa chỉ contract của collection trên blockchain. \\ \hline
created\_at & TIMESTAMP & Thời điểm tạo bản ghi. \\ \hline
updated\_at & TIMESTAMP & Thời điểm cập nhật gần nhất. \\ \hline
image\_url & TEXT & URL ảnh đại diện collection. \\ \hline
symbol & VARCHAR(30) & Ký hiệu (symbol) của collection (ví dụ: BAYC). \\ \hline
royalty\_percent & NUMERIC(5,2) & Phần trăm royalty áp dụng cho collection (ví dụ: 2.50). \\ \hline

\caption{Chi tiết bảng Collection}
\label{tab:collection-table}
\end{longtable}

Đặc tả bảng NFT:

\begin{longtable}{|l|l|p{9cm}|}
\hline
\textbf{Trường} & \textbf{Kiểu dữ liệu} & \textbf{Mô tả} \\ \hline
\endfirsthead
\hline
\textbf{Trường} & \textbf{Kiểu dữ liệu} & \textbf{Mô tả} \\ \hline
\endhead

id & BIGINT & Khóa chính (PK) định danh NFT trong DB. \\ \hline
name & VARCHAR(150) & Tên NFT. \\ \hline
description & TEXT & Mô tả NFT. \\ \hline
price & NUMERIC(38,0) & Giá hiện tại (có thể lưu theo wei hoặc theo đơn vị chuẩn tùy thiết kế). \\ \hline
image\_url & TEXT & URL ảnh NFT (S3/IPFS/HTTP...). \\ \hline
token\_id & BIGINT & TokenId on-chain của NFT. \\ \hline
is\_for\_sale & BOOLEAN & Trạng thái đang được niêm yết bán hay không. \\ \hline
contract\_address & VARCHAR(42) & Địa chỉ contract chứa NFT. \\ \hline
owner\_id & BIGINT & FK \(\rightarrow\) users.id, owner hiện tại. \\ \hline
collection\_id & BIGINT & FK \(\rightarrow\) collections.id, NFT thuộc collection nào. \\ \hline
created\_at & TIMESTAMP & Thời điểm tạo bản ghi. \\ \hline
updated\_at & TIMESTAMP & Thời điểm cập nhật gần nhất. \\ \hline
creator\_id & BIGINT & FK \(\rightarrow\) users.id, người tạo/mint NFT. \\ \hline
royalty\_percent & NUMERIC(5,2) & Royalty cho NFT. \\ \hline

\caption{Chi tiết bảng NFT}
\label{tab:nft-table}
\end{longtable}

Đặc tả bảng NFT event:

\begin{longtable}{|l|l|p{9cm}|}
\hline
\textbf{Trường} & \textbf{Kiểu dữ liệu} & \textbf{Mô tả} \\ \hline
\endfirsthead
\hline
\textbf{Trường} & \textbf{Kiểu dữ liệu} & \textbf{Mô tả} \\ \hline
\endhead

id & BIGINT & Khóa chính (PK) định danh sự kiện NFT. \\ \hline
nft\_id & BIGINT & FK \(\rightarrow\) nfts.id, NFT liên quan tới sự kiện. \\ \hline
event\_type & VARCHAR(50) & Loại sự kiện (mint/list/buy/transfer/cancel...). \\ \hline
from\_address & VARCHAR(42) & Địa chỉ ví nguồn (người gửi/seller). \\ \hline
to\_address & VARCHAR(42) & Địa chỉ ví đích (người nhận/buyer). \\ \hline
price\_wei & NUMERIC(38,0) & Giá giao dịch theo wei (nếu có). \\ \hline
tx\_hash & VARCHAR(66) & Transaction hash trên blockchain. \\ \hline
log\_index & INT & Log index trong transaction (phục vụ đối chiếu event log). \\ \hline
block\_number & BIGINT & Số block chứa transaction. \\ \hline
block\_timestamp & TIMESTAMP & Thời điểm block được ghi nhận. \\ \hline
chain\_id & INT & Mã mạng blockchain phát sinh sự kiện. \\ \hline
created\_at & TIMESTAMP & Thời điểm hệ thống lưu sự kiện vào DB. \\ \hline
platform\_fee\_wei & NUMERIC(38,0) & Phí nền tảng thu được (tính theo wei). \\ \hline
royalty\_amount\_wei & NUMERIC(38,0) & Số tiền royalty trả cho creator (theo wei). \\ \hline
royalty\_receiver & VARCHAR(42) & Địa chỉ nhận royalty. \\ \hline

\caption{Chi tiết bảng NFT event}
\label{tab:nft-event-table}
\end{longtable}

\section{Xây dựng ứng dụng}
\subsection{Thư viện và công cụ sử dụng}
Các công cụ được sử dụng trong quá trình phát triển sản phẩm được liệt kê dưới bảng \ref{tab:tools}:

\begin{table}[!htbp]
\centering
\small
\setlength{\tabcolsep}{3pt}
\renewcommand{\arraystretch}{1.15}

\begin{tabular}{|c|p{3.2cm}|p{3.6cm}|p{2.0cm}|p{4.8cm}|}
\hline
\textbf{STT} & \textbf{Mục đích} & \textbf{Tên công cụ} & \textbf{Phiên bản} & \textbf{URL} \\ \hline

1 & Ngôn ngữ + runtime backend & Node.js (LTS) + TypeScript & 24.12.0; 5.9.3 &
\url{https://nodejs.org/}; \url{https://www.typescriptlang.org/} \\ \hline

2 & Framework backend API & NestJS & 11.1.10 &
\url{https://nestjs.com/} \\ \hline

3 & Framework frontend web & Next.js & 16.1.1 &
\url{https://nextjs.org/} \\ \hline

4 & Hệ quản trị CSDL (SQL) & PostgreSQL & 18.1 &
\url{https://www.postgresql.org/} \\ \hline

5 & Cache / tăng tốc truy vấn & Redis (Open Source) & 8.4.0 &
\url{https://redis.io/} \\ \hline

6 & IDE phát triển & Visual Studio Code & 1.107.1 &
\url{https://code.visualstudio.com/} \\ \hline

7 & Triển khai/vận hành trên AWS & AWS CLI (v2) & 2.32.24 &
\url{https://aws.amazon.com/cli/} \\ \hline

\end{tabular}

\caption{Các công cụ chính sử dụng trong quá trình phát triển hệ thống}
\label{tab:tools}
\end{table}

\FloatBarrier

\subsection{Kết quả đạt được(cần cập nhật lại bảng chỉ số thống kê)}

Trong quá trình thực hiện đồ án, em đã xây dựng được một hệ thống NFT Marketplace có thể vận hành và trình diễn đầy đủ các luồng chức năng chính. Kết quả đạt được không chỉ dừng ở việc hoàn thiện mã nguồn, mà còn được đóng gói thành các sản phẩm cụ thể để thuận tiện cho việc triển khai, kiểm thử và demo. Các sản phẩm đóng gói bao gồm: ứng dụng giao diện người dùng (frontend) giúp người dùng thao tác tạo/bán/mua và khám phá NFT; dịch vụ backend API xử lý nghiệp vụ, quản lý dữ liệu và đồng bộ trạng thái sau giao dịch; smart contract triển khai các chức năng on-chain liên quan đến NFT; ngoài ra kèm theo các script cấu hình/migration phục vụ khởi tạo dữ liệu. Việc đóng gói theo từng thành phần giúp hệ thống có cấu trúc rõ ràng, dễ quản lý và có thể mở rộng trong tương lai.

\begin{table}[!htbp]
\centering
\small
\setlength{\tabcolsep}{4pt}
\renewcommand{\arraystretch}{1.15}
\begin{tabular}{|p{6.2cm}|c|c|c|c|}
\hline
\textbf{Chỉ số thống kê} & \textbf{Frontend} & \textbf{Backend} & \textbf{Smart contract} & \textbf{Tổng} \\ \hline
Số dòng code (LOC) & \textit{...} & \textit{...} & \textit{...} & \textit{...} \\ \hline
Số thư mục/module chính & \textit{...} & \textit{...} & \textit{...} & \textit{...} \\ \hline
Số file mã nguồn & \textit{...} & \textit{...} & \textit{...} & \textit{...} \\ \hline
Số lớp (class) / thành phần chính & \textit{...} & \textit{...} & \textit{...} & \textit{...} \\ \hline
Dung lượng toàn bộ mã nguồn (MB) & \textit{...} & \textit{...} & \textit{...} & \textit{...} \\ \hline
Dung lượng sản phẩm sau build (MB) & \textit{...} & \textit{...} & \textit{...} & \textit{...} \\ \hline
\end{tabular}
\caption{Thống kê quy mô mã nguồn và sản phẩm đóng gói}
\label{tab:app-statistics}
\end{table}

\subsection{Minh họa các chức năng chính}
Sinh viên lựa chọn và đưa ra màn hình cho các chức năng chính, quan trọng, và thú vị nhất. Mỗi giao diện cần phải có lời giải thích ngắn gọn. Khi giải thích, sinh viên có thể kết hợp với các chú thích ở trong hình ảnh giao diện.

\section{Kiểm thử}

\subsection{Mục tiêu và phạm vi kiểm thử}
Phần này tập trung thiết kế và thực hiện kiểm thử cho các chức năng quan trọng nhất của hệ thống NFT Marketplace: đăng nhập kết nối ví; tạo NFT; niêm yết và mua/bán NFT.

\subsection{Kỹ thuật kiểm thử đã sử dụng}
Trong quá trình thiết kế test case, em sử dụng kết hợp các kỹ thuật sau:
\begin{itemize}
  \item \textbf{Kiểm thử hộp đen (Black-box testing):} thiết kế test case dựa trên đầu vào/đầu ra và hành vi mong đợi theo yêu cầu.
  \item \textbf{Phân lớp tương đương (Equivalence Partitioning):} chia dữ liệu đầu vào thành nhóm hợp lệ/không hợp lệ để giảm số lượng test case nhưng vẫn đảm bảo bao phủ.
  \item \textbf{Giá trị biên (Boundary Value Analysis):} kiểm thử tại các ngưỡng quan trọng (ví dụ giá = 0, giá rất nhỏ, độ dài chuỗi tối đa,\dots).
  \item \textbf{Kiểm thử theo luồng (Scenario/Flow testing):} kiểm thử theo chuỗi hành động thực tế (login $\rightarrow$ tạo NFT $\rightarrow$ list $\rightarrow$ buy $\rightarrow$ cập nhật).
  \item \textbf{Kiểm thử API/kiểm thử tích hợp (API \& Integration testing):} gọi API và đối chiếu dữ liệu phát sinh trong cơ sở dữ liệu.
\end{itemize}

\subsection{Thiết kế các trường hợp kiểm thử}

\subsubsection{Chức năng 1: Đăng nhập bằng ví (SIWE)}
Người dùng lấy nonce, ký thông điệp SIWE bằng ví, backend xác thực chữ ký và cấp access token.

\begingroup
\footnotesize
\setlength{\tabcolsep}{2pt}
\renewcommand{\arraystretch}{1.15}

\begin{longtable}{|
>{\raggedright\arraybackslash}p{2.1cm}|
>{\raggedright\arraybackslash}p{2.5cm}|
>{\raggedright\arraybackslash}p{2.7cm}|
>{\raggedright\arraybackslash}p{3.6cm}|
>{\raggedright\arraybackslash}p{3.6cm}|}
\hline
\textbf{Mã TC} & \textbf{Mục tiêu} & \textbf{Dữ liệu vào} & \textbf{Các bước thực hiện} & \textbf{Kết quả mong đợi} \\ \hline
\endfirsthead
\hline
\textbf{Mã TC} & \textbf{Mục tiêu} & \textbf{Dữ liệu vào} & \textbf{Các bước thực hiện} & \textbf{Kết quả mong đợi} \\ \hline
\endhead

TC-LOGIN-01 & Đăng nhập thành công &
wallet hợp lệ, message hợp lệ, signature hợp lệ &
(1) Lấy nonce (2) Ký SIWE (3) Gửi verify/login &
Trả token; user được tạo mới(nếu chưa tồn tại trong hệ thống). \\ \hline

TC-LOGIN-02 & Sai chữ ký &
signature không khớp message &
Gửi verify/login với signature sai &
Trả lỗi và không cấp token. \\ \hline

TC-LOGIN-03 & Nonce đã dùng/hết hạn &
nonce cũ hoặc bị dùng lại &
Dùng lại nonce và gửi verify/login &
Bị từ chối và yêu cầu lấy nonce mới. \\ \hline

TC-LOGIN-04 & Sai chain/domain &
chainId hoặc domain không đúng cấu hình &
Gửi verify/login với chain/domain sai &
Bị từ chối xác thực và trả lỗi. \\ \hline

TC-LOGIN-05 & Wallet sai định dạng &
wallet address sai chuẩn &
Gửi verify/login với địa chỉ ví sai &
Validate thất bại và trả lỗi. \\ \hline

\caption{Thiết kế test case cho chức năng đăng nhập bằng ví (SIWE)}
\label{tab:test-login}
\end{longtable}
\endgroup


\subsubsection{Chức năng 2: Tạo NFT}
Người dùng nhập thông tin NFT (tên, mô tả, ảnh/metadata), backend lưu dữ liệu NFT và liên kết vào collection.

\begingroup
\footnotesize
\setlength{\tabcolsep}{2pt}
\renewcommand{\arraystretch}{1.15}

\begin{longtable}{|
>{\raggedright\arraybackslash}p{2.1cm}|
>{\raggedright\arraybackslash}p{2.5cm}|
>{\raggedright\arraybackslash}p{2.7cm}|
>{\raggedright\arraybackslash}p{3.6cm}|
>{\raggedright\arraybackslash}p{3.6cm}|}
\hline
\textbf{Mã TC} & \textbf{Mục tiêu} & \textbf{Dữ liệu vào} & \textbf{Các bước thực hiện} & \textbf{Kết quả mong đợi} \\ \hline
\endfirsthead
\hline
\textbf{Mã TC} & \textbf{Mục tiêu} & \textbf{Dữ liệu vào} & \textbf{Các bước thực hiện} & \textbf{Kết quả mong đợi} \\ \hline
\endhead

TC-NFT-01 & Tạo NFT hợp lệ &
dữ liệu NFT hợp lệ &
Gọi POST /nfts &
Tạo thành công, trả về dữ liệu đúng, DB có bản ghi mới. \\ \hline

TC-NFT-02 & Thiếu trường bắt buộc &
name rỗng hoặc thiếu &
Gọi POST /nfts &
Trả lỗi, không tạo bản ghi. \\ \hline

TC-NFT-03 & Collection không tồn tại &
collectionId sai &
Gọi POST /nfts &
Trả lỗi, không tạo NFT. \\ \hline

TC-NFT-04 & Ảnh/URL sai định dạng &
imageUrl không hợp lệ &
Gọi POST /nfts &
Trả lỗi, thông báo rõ trường sai. \\ \hline

TC-NFT-05 & Giá trị biên (tên) &
name quá dài hoặc vượt giới hạn &
Gọi POST /nfts &
Trả lỗi validation và yêu cầu user nhập lại. \\ \hline

\caption{Thiết kế test case cho chức năng tạo NFT}
\label{tab:test-create-nft}
\end{longtable}
\endgroup


\subsubsection{Chức năng 3: Niêm yết bán và cập nhật trạng thái sau giao dịch}
User niêm yết NFT với giá bán; sau giao dịch mua trên ví, frontend gọi API để backend cập nhật dữ liệu dựa theo transaction event.

\begingroup
\footnotesize
\setlength{\tabcolsep}{2pt}
\renewcommand{\arraystretch}{1.15}

\begin{longtable}{|
>{\raggedright\arraybackslash}p{2.1cm}|
>{\raggedright\arraybackslash}p{2.5cm}|
>{\raggedright\arraybackslash}p{2.7cm}|
>{\raggedright\arraybackslash}p{3.6cm}|
>{\raggedright\arraybackslash}p{3.6cm}|}
\hline
\textbf{Mã TC} & \textbf{Mục tiêu} & \textbf{Dữ liệu vào} & \textbf{Các bước thực hiện} & \textbf{Kết quả mong đợi} \\ \hline
\endfirsthead
\hline
\textbf{Mã TC} & \textbf{Mục tiêu} & \textbf{Dữ liệu vào} & \textbf{Các bước thực hiện} & \textbf{Kết quả mong đợi} \\ \hline
\endhead

TC-LIST-01 & List NFT thành công &
nft thuộc owner hiện tại, price $>$ 0 &
Gọi POST /listings &
Chuyển trạng thái NFT listing sang active và cập nhật giá bán. \\ \hline

TC-LIST-02 & Không phải chủ sở hữu &
nft không thuộc user hiện tại &
Gọi POST /listings &
Trả lỗi và không tạo listing. \\ \hline

TC-LIST-03 & Giá bán không hợp lệ (biên) &
price = 0 hoặc price $<$ 0 &
Gọi POST /listings &
Trả lỗi và không tạo listing. \\ \hline

TC-BUY-01 & Xác nhận mua thành công &
txHash hợp lệ, buyer/seller hợp lệ &
Gọi POST /transactions/confirm &
Cập nhật thành công owner sang buyer, tạo bản ghi history/event tương ứng. \\ \hline

TC-BUY-02 & Trùng txHash (idempotent) &
txHash đã tồn tại trong DB &
Gọi lại confirm với cùng txHash &
Không tạo trùng; trả kết quả an toàn khi retry (idempotent). \\ \hline

TC-CANCEL-01 & Hủy listing &
listing đang active &
Gọi POST /listings/cancel &
Chuyển trạng thái listing sang unactive thành công. \\ \hline

\caption{Thiết kế test case cho chức năng niêm yết và cập nhật trạng thái sau giao dịch}
\label{tab:test-list-buy}
\end{longtable}
\endgroup

\section{Triển khai(khi nào deploy có cấu hình server thì cập nhật vào đây)}

\end{document}
